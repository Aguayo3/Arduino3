\section{Marco teórico}
El presente laboratorio aborda el uso de periféricos y técnicas de control digital aplicadas en sistemas embebidos, específicamente el manejo de un joystick, 
una matriz de LEDs 8x8 y un módulo LCD1602 con interfaz I2C. Estos elementos permiten implementar aplicaciones interactivas que combinan entrada y salida de datos,
reforzando conceptos de electrónica digital y programación en microcontroladores.
\subsection{Arduino Uno}
El Arduino Uno es una placa de desarrollo basada en el microcontrolador ATmega328P de $8$ bits, este opera a una
frecuencia de $\SI{16}{\mega \hertz}$. Posee $\SI{1}{\kilo \byte}$ de memoria EEPROM, $\SI{2}{\kilo \byte}$ de memoria 
SRAM y $\SI{32}{\kilo \byte}$ de memoria flash \cite{medina-arduino}. Este microcontrolador es ampliamente utilizada en proyectos educativos,
de prototipado y aplicaciones de electrónica digital.

La placa cuenta con 14 pines digitales de entrada/salida 
(de los cuales 6 pueden usarse como salidas PWM),
6 entradas analógicas utilizando un conversor análogo-digital de 10 bits de resolución, 
un puerto USB para comunicación y alimentación, 
además de un regulador de voltaje que permite conectarla a fuentes externas. 

Una de las características principales del Arduino Uno es su capacidad para interactuar con el entorno físico,
permitiendo leer señales analógicas o digitales provenientes de sensores, procesarlas en el microcontrolador 
y generar respuestas mediante actuadores, como motores o LEDs.


\subsection{Matriz LED 8x8}

Una matriz LED es un conjunto de diodos emisores de luz organizados en filas y columnas, que se controla
mediante técnicas de multiplexación para reducir la cantidad de pines necesarios. En este laboratorio, se
busca evitar el uso directo de las salidas digitales del microcontrolador, empleando registros de
desplazamiento que permiten convertir datos seriales en paralelos. Esto optimiza el uso de recursos del
Arduino y mejora la escalabilidad del sistema.

\subsection{Registros de desplazamiento}

Los registros de desplazamiento son circuitos digitales capaces de almacenar y desplazar información. Se
utilizan ampliamente para expandir el número de salidas disponibles en un microcontrolador sin aumentar el
número de pines empleados. Dentro de este laboratorio se utilizaron dos integrados:
\begin{itemize}
    \item 74HC595: Es un registro de desplazamiento de 8 bits con salida en paralelo y entrada en serie. Incluye
    una compuerta de almacenamiento (latch) que permite mantener los datos estables en las salidas
    mientras se cargan nuevos valores. Es uno de los integrados más usados para controlar LEDs, displays y otros periféricos con pocos pines del microcontrolador.
    \item 74HC95: Es un registro de desplazamiento de 4 bits, el cual puede configurarse para desplazar datos en
    serie o entregar salidas en paralelo.
\end{itemize}

\subsection{Joystick}

El joystick analógico funciona mediante dos potenciómetros dispuestos ortogonalmente, que permiten
obtener variaciones de voltaje en los ejes X e Y. Estas señales analógicas son leídas por el conversor A/D del
microcontrolador, lo que posibilita interpretar movimientos direccionales. Además, muchos joysticks
incluyen un botón integrado que puede utilizarse para iniciar o reiniciar una aplicación. En este laboratorio,
el joystick cumple el rol de interfaz principal de control, permitiendo desplazar un "cazador" dentro de la matriz LED.

\subsection{Pantalla LCD1602 con I2C}
La pantalla LCD1602 es un display alfanumérico de 2 filas por 16 caracteres, ampliamente usado en sistemas embebidos por su bajo costo y 
facilidad de integración. La incorporación del módulo I2C simplifica la conexión al microcontrolador, requiriendo solo dos líneas de
comunicación (SDA y SCL) en lugar de múltiples pines de control.
