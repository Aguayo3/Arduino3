\section{Marco teórico}
\subsection{Arduino Uno}
El Arduino Uno es una placa de desarrollo basada en el microcontrolador ATmega328P de $8$ bits, este opera a una
frecuencia de $\SI{16}{\mega \hertz}$. Posee $\SI{1}{\kilo \byte}$ de memoria EEPROM, $\SI{2}{\kilo \byte}$ de memoria 
SRAM y $\SI{32}{\kilo \byte}$ de memoria flash \cite{medina-arduino}. Este microcontrolador es ampliamente utilizada en proyectos educativos,
de prototipado y aplicaciones de electrónica digital.


La placa cuenta con 14 pines digitales de entrada/salida 
(de los cuales 6 pueden usarse como salidas PWM),
6 entradas analógicas utilizando un conversor análogo-digital de 10 bits de resolución, 
un puerto USB para comunicación y alimentación, 
además de un regulador de voltaje que permite conectarla a fuentes externas. 


Una de las características principales del Arduino Uno es su capacidad para interactuar con el entorno físico,
permitiendo leer señales analógicas o digitales provenientes de sensores, procesarlas en el microcontrolador 
y generar respuestas mediante actuadores, como motores o LEDs.

\subsection{Matriz MAX7219}
La matriz LED 8×8 con controlador MAX7219 es un módulo de visualización basado en 64 diodos organizados en filas y columnas, gestionados por un
circuito integrado que simplifica el encendido y apagado de cada punto mediante multiplexación. El MAX7219 permite comunicación serial con 
microcontroladores, reduciendo el uso de pines y facilitando la implementación de aplicaciones como displays de texto, números o patrones gráficos