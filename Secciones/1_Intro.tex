\section{Introducción}

En este laboratorio se emplea un Arduino Uno junto con un módulo de visualización LED basado en el MAX7219,
utilizando la librería MD\_MAX72XX para facilitar la comunicación y control de los LEDs. Se utiliza un botón 
azul para incrementar el valor mostrado de uno en uno y un botón rojo para disminuirlo de la misma manera. 
El objetivo del experimento es comprender cómo los botones interactúan con el microcontrolador, 
cómo se gestionan las entradas y salidas digitales, y cómo se controla un módulo externo mediante SPI y librerías especializadas, 
reforzando así conceptos fundamentales de programación y manejo de dispositivos electrónicos.