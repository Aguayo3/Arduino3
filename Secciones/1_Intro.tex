\section{Introducción}
En el uso de sistemas digitales y los microcontroladores, el manejo de periféricos externos constituye
una habilidad fundamental para el diseño de aplicaciones interactivas. El presente laboratorio tiene como
objetivo implementar un sistema que integre distintos dispositivos de entrada y salida, específicamente un
joystick, una matriz LED de 8x8 y una pantalla LCD1602 con interfaz I2C.


La actividad consiste en diseñar un circuito controlador que permita gestionar la matriz LED utilizando
registros de desplazamiento. A través del joystick, el usuario puede interactuar con un juego en el que debe mover un “cazador” hasta un
objetivo, evitando obstáculos y limitaciones de la matriz. Paralelamente, la pantalla LCD se emplea para
mostrar mensajes de estado, tiempos de juego y resultados obtenidos, brindando un sistema completo de retroalimentación visual.


Este laboratorio busca reforzar los conocimientos adquiridos en experiencias previas, como el control de una
matriz LED mediante el integrado MAX7219, pero ahora incorporando nuevas estrategias de control y
comunicación. Además, fomenta la comprensión práctica de conceptos como multiplexación, registros de
desplazamiento, comunicación I2C y conversión analógica-digital, los cuales son esenciales en el desarrollo
de sistemas electrónicos modernos.