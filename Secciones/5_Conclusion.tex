\section{Conclusión}

En esta experiencia de  laboratorio se logró de forma exitosa el uso de un microcontrolador Arduino Uno
y un módulo de matriz de LED MAX7219. El objetivo principal fue comprender el manejo de las entradas y
salidas digitales, así como el control de un módulo externo mediante el protocolo SPI y librerías especializadas.
Mediante la implementación de un circuito con dos pulsadores—uno para incrementar y otro para decrementar
un valor, se aplicaron de manera práctica estos conceptos. Se logró implementar y mostrar en la matriz de
LED la lógica de conteo en base 16 (hexadecimal), de 0 a F. La utilización de la librería MD\_MAX72xx
simplificó enormemente la comunicación con el módulo, y las funciones mx.clear() y mx.setChar() fueron
clave para controlar la visualización de los caracteres.
La fase de implementación física del prototipo en una protoboard confirmó el correcto funcionamiento del
diseño. El código desarrollado, que incluyó un método antirrebote para cada botón, demostró ser una
solución efectiva para garantizar que cada pulsación fuera registrada una única vez.
